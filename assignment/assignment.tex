\documentclass[jobname=project, 10pt]{article}
\usepackage{hyperref}
\usepackage{graphicx}
\usepackage{fancyhdr}
\usepackage[margin=1in, top=2in, bottom=1.25in, headheight=3cm, a4paper]{geometry}
\usepackage{enumitem}
\usepackage[scriptsize]{caption}
\usepackage{subcaption}
\usepackage{xcolor}
\usepackage{float}
\usepackage{amsmath}
\usepackage{amssymb}
\usepackage{tikz}
\usetikzlibrary{patterns}

% DERIVATIVE OPERATORS 
\newcommand{\D}{D}              % upper-case 'D" for total derivative 
\newcommand{\p}{\partial}       % round 'del'  for partial differentiation
\renewcommand{\d}{\mathrm{d}}   % roman d for integrals
\renewcommand{\vec}[1]{\mathbf{#1}}
% italic d is d (no command)    % regular differentiation of scalar function 

\pagestyle{fancy}
\fancyhf{}
\fancyhead[L]{\includegraphics[width=5cm]{fu_logo_web.png}}  % Logo à gauche
\fancyhead[C]{Numerics practical}  % Au milieu
\fancyhead[R]{\shortstack{Theoretical Meteorology I \\ Cedrick Ansorge \& Sally Issa \\ \today}}  % Auteur sur une ligne, date sur une ligne
\renewcommand{\headrulewidth}{0.4pt}  % Ligne horizontale
\newcommand{\task}[2]{\section{#1 \hfill \normalsize\normalfont \textcolor{gray}{#2\,P}}\addtocounter{ptot}{#2}}
\newcommand{\hint}[1]{
  \phantom{AA}~\\[1.0em]
  \colorbox{gray!30}{
    \begin{minipage}{.92\textwidth}
      \vspace{-1em}\centering{\fcolorbox{black}{white}{\emph{Hint for solution}}} 
      {\footnotesize #1}
    \end{minipage} 
  }
}
\newcounter{ptot}
\setlength{\parskip}{0em}
\begin{document}
~\\[-3em]
\phantom{AA}\hfill \textcolor{red}{\textbf{Discussion Feb 16$^\mathrm{th}$, 4pm (neuer H\"orsaal)}} \\
This practical gives a practical perspective on some aspects of the quasi-geostrophic approximation.
We will use the Quasi-geostrophic barotropic vorticity with and without a forcing and the two-layer model
to reflect on some of the theoretical findings derived during the semester.
%
\par
% 
We devote the final exercise on Feb 16$^{th}$ to a discussion of your results. Please hand in your code and a small expos\'e consisting of some figures showing results from the model prior to this discussion. We expect a brief discussion / presentation of the expos\'e during this exercise. 

\task{Preliminaries - model infrastructure}{3}

You can use the model infrastructure provided for python that is provided on the whiteboard. (Alternatively, you may develop an infrastructure with similar capabilities in a programming/scripting language of your choice.) \\ 

\noindent\textbf{Make sure you understand the following aspects before you get started to implement any other aspect of this assignment:}
\begin{itemize}
\item How is the model controlled (how does it learn about its input / parameters / configurations?) ?
\item Which classes exist and what is there purpose?
  \begin{enumerate}
    \setlength{\itemsep}{1.5em}
  \item \underline{\phantom{CFQGCFQGCFQGCFQG}} -- \underline{\phantom{CFQGCFQGCFQGCFQG}}
  \item \underline{\phantom{CFQGCFQGCFQGCFQG}} -- \underline{\phantom{CFQGCFQGCFQGCFQG}}
  \item \underline{\phantom{CFQGCFQGCFQGCFQG}} -- \underline{\phantom{CFQGCFQGCFQGCFQG}}
  \item \underline{\phantom{CFQGCFQGCFQGCFQG}} -- \underline{\phantom{CFQGCFQGCFQGCFQG}}
  \end{enumerate}
\item What time integration scheme is used? In wich class and routine is it implemented? 
  \begin{itemize} 
    \setlength{\itemsep}{1.5em}
  \item[Scheme] \underline{\hspace{6cm}}
  \item[Class]  \underline{\hspace{6cm}}
  \item[Routine] \underline{\hspace{6cm}}
  \end{itemize}
    
\item What kind of output is generated to the following streams/files 
  \begin{itemize}
    \setlength{\itemsep}{1.5em}

  \item[stdout] \underline{\hspace{6cm}}
  \item[*.log] \underline{\hspace{6cm}} 
  \item[*.iter] \underline{\hspace{6cm}}
  \item[*.nc] \underline{\hspace{6cm}} 
  \end{itemize}

\end{itemize} 

\task{Time integration}{3} 
Check the time integration scheme implemented in the class \texttt{INTEGRATOR}. Setting \texttt{rhs} \footnote{Note that with \texttt{rhs}, we are passing a function as argument, i.e. we are using the concept of function pointers} to \texttt{exponential} (a member of the class \texttt{integrator}) evaluates to the input, i.e. we solve the equation
\[\frac{d x}{d t} = x \Rightarrow \tilde{x}(t) = \frac{x_0}{e^{t_0}} e^t. \]

\begin{enumerate}[label=\alph*)]
\item What is the expected result  for $t_0=0$ and $x_0=1$ when integrating up to $t=2$? Check for $\Delta t=1$!
\item Varying the time step $\tau$ of the integration, check the order of the time integration scheme
  and produce a log-log plot of the error $x_{num} - \tilde{x}$ as a function of $\tau$.
\item What is the order of the numerical error for the time integration scheme used?
\item Why does the numerical error reach a limit as $\tau \rightarrow 0$? 
\end{enumerate}

\task{Rossby waves}{15}
For this task, you need to use the \texttt{rhs} dummy routine \texttt{rossby} and the section \texttt{Rossby} in \texttt{qg.ini}
\begin{enumerate}[label=\alph*)]
\item Implement the barotropic vorticity equation
  \[ \left[\frac{\p}{\p t} + \left(U_j + {u'}_j\right) \frac{\p}{\p x_j}\right] \nabla^2 \zeta - \beta v = 0 \]
\item Investigate the behavior of waves for a \texttt{harmonic} initial perturbation! Can you prodcue standing waves? Give the parmaeter setting!
\item Investigate the behavior of a \texttt{singular} perturbation!
  \begin{itemize}
  \item What do you observe?
  \item How does the behavior differ from a harmonic initial condition?
  \item Can you explain this?
  \end{itemize}
\end{enumerate}

\task{Two-layer model}{20}
\begin{enumerate}[label=\alph*)]
\item Implement the two-layer model using a prognostic equation for $\zeta$ at the $250$ and $750$ millibar level (assuming $p_\text{sfc}=const.=1000\,\mathrm{hPa}$)!
\item Investigate the growth rate of a baroclinic instability!
\item Analyze the terms of the Lorenz energy cycle!


\end{document}